% Created 2022-11-14 Mon 10:28
% Intended LaTeX compiler: pdflatex
\documentclass[letterpaper,parskip=half]{scrartcl}
\usepackage[utf8]{inputenc}
\usepackage[T1]{fontenc}
\usepackage{graphicx}
\usepackage{longtable}
\usepackage{wrapfig}
\usepackage{rotating}
\usepackage[normalem]{ulem}
\usepackage{amsmath}
\usepackage{amssymb}
\usepackage{capt-of}
\usepackage{hyperref}
\usepackage{braket}
\usepackage{mlmodern}
% \usepackage[euler-digits,euler-hat-accent]{eulervm}
\usepackage[T1]{fontenc}

% \KOMAoptions{parskip=half}

\usepackage[margin=2.5cm,includehead=true,includefoot=true,centering]{geometry}

\usepackage{stmaryrd,csquotes}
\usepackage{amsthm}
\usepackage{amsmath}
\usepackage{amsfonts}
\usepackage{mathtools}
\usepackage[linguistics]{forest}

\usepackage{tikz}
\usepackage{nicematrix}

\usepackage{awesomebox}

\usepackage{xparse}
\usepackage{braket}
\usepackage{pifont}

\usepackage{todonotes}

\usepackage{appendix}

\usepackage{tcolorbox}

\usepackage{colortbl}

\usepackage{booktabs}

\usepackage[normalem]{ulem}

\usepackage[
  backend=biber
, bibstyle=biblatex-sp-unified
, citestyle=sp-authoryear-comp
, url=true
, doi=false
, bibencoding=utf8]{biblatex}

\usepackage{xpatch}
\makeatletter
\xpatchcmd{\@maketitle}{\begin{center}}{\begin{flushleft}}{}{}
\xpatchcmd{\@maketitle}{\end{center}}{\end{flushleft}}{}{}
\xpatchcmd{\@maketitle}{\begin{tabular}[t]{c}}{\begin{tabular}[t]{@{}l@{}}}{}{}
\makeatother

\NewDocumentCommand\eval{sO{}O{}m}{%
  \IfBooleanTF#1
  {\ensuremath{\left\llbracket{#4}\right\rrbracket^{#2}_{#3}}}
  {\ensuremath{\left\llbracket\text{#4}\right\rrbracket^{#2}_{#3}}}
}

\NewDocumentCommand{\sub}{m}{\textsubscript{#1}}
\NewDocumentCommand{\supscr}{m}{\textsuperscript{#1}}

\NewDocumentCommand\fap{}{\ensuremath{\mathbin{/\!/}}}
\NewDocumentCommand\bap{}{\ensuremath{\mathbin{\backslash\!\backslash}}}

\theoremstyle{definition}
\newtheorem{definition}{Definition}[section]

\theoremstyle{fact}
\newtheorem{fact}{Fact}[section]

\usepackage{float}

\usepackage{hyperref}

\usepackage{gb4e}
\author{Patrick D. Elliott}
\date{\today}
\title{Plurality and collective predication cont.\\\medskip
\large Handout 3}
\hypersetup{
 pdfauthor={Patrick D. Elliott},
 pdftitle={Plurality and collective predication cont.},
 pdfkeywords={},
 pdfsubject={},
 pdfcreator={Emacs 28.2 (Org mode 9.5.5)}, 
 pdflang={English}}
\usepackage{biblatex}
\addbibresource{/home/patrl/repos/bibliography/master.bib}
\addbibresource{~/repos/bibliography/master.bib}
\begin{document}

\maketitle
\tableofcontents


\section{Reading}
\label{sec:orgab63bcb}

\begin{itemize}
\item \autocite{Champollion2016}
\end{itemize}

\section{Recap: distributive and collective predication}
\label{sec:orgdfc04c0}

\subsection{Distributive and collective predication}
\label{sec:org69f0dfa}

Generalized conjunction (\(\sqcap\)) and generalized disjunction (\(\sqcup\)), which are recursive generalizations of boolean conjunction/disjunction respectively, predict \textit{distributive inferences} across a broad variety of example sentences.

\begin{exe}
\ex \textbf{Generalized conjunction:}\\[0pt]
 \(P_\tau \sqcap Q_\tau = \begin{cases}
P \wedge Q&\text{if }\tau = T\\
\lambda x\,.\,P(x) \sqcap_\rho Q(x)&\text{if }\tau = \sigma \rightarrow \rho  
\end{cases}\)
\label{org3053c4c}
\end{exe}

\begin{exe}
\ex Mary and John slept.\\[0pt]
\(\iff\) Mary slept and John slept.
\ex Most women or most men are tall.\\[0pt]
\(\iff\) Most women are tall or most men are tall.
\ex Neither the milkman nor the postman arrived.\\[0pt]
\(\iff\) Neither did the milkman arrive, nor did the postman.
\label{org1b3d0ee}
\end{exe}

\subsection{Montague Lift}
\label{sec:orgdaaa2ea}

\emph{Montague lift} converts expressions such as \textbf{Mary} and \textbf{John} into expressions with a boolean type, allowing them to be subject to generalized conjunction.

\begin{exe}
\ex \(\mathbf{Mary}^\uparrow := \lambda P_{_{ET}}\,.\,P(\mathbf{Mary})\)
\label{orgf4c2814}
\end{exe}

\begin{exe}
\ex \(\mathbf{Mary}^\uparrow \sqcap \mathbf{John}^\uparrow(\mathbf{sleep})\)
\ex \(((\lambda P\,.\,P(\mathbf{Mary})) \sqcap (\lambda P\,.\,P(\mathbf{John})))(\mathbf{sleep})\)
\ex \((\lambda P\,.\,P(\mathbf{Mary}) \wedge P(\mathbf{John}))(\mathbf{sleep})\)
\ex \(\mathbf{sleep}(\mathbf{Mary}) \wedge \mathbf{sleep}(\mathbf{John})\)
\label{orge9548a2}
\end{exe}

\subsection{Collective predication}
\label{sec:orge1d507b}

Verbs:

\begin{exe}
\ex Lexically collective: \emph{gather}, \emph{disperse}, \emph{meet} (when used intransitively), \emph{outnumber} (both arguments).
\ex Lexically ``mixed'' collective/distributive: the subject argument of \emph{write}, \emph{lift}, \emph{eat}, and \emph{carry}.
\ex \emph{together} modification: \emph{sing together}, \emph{live together}, \emph{write NP together}.
\ex \emph{between them} modification: \emph{grade over 200 papers between them}, \emph{ate 15 pizzas between them}.
\ex Reciprocal modification: \emph{like each other}, \emph{look at one another}.
\label{orgc8683a2}
\end{exe}

Adjectives:

\begin{exe}
\ex Lexically collective: \emph{numerous}, \emph{similar}, \emph{alike}, \emph{parallel}, \emph{antagonistic}, \emph{equivalent}, \emph{neighboring}.
\ex \emph{together} modification: \emph{happy together}, \emph{irritating together}.
\ex Reciprocal modification: \emph{nice to each other}, \emph{fond of one another}.
\label{orgd4ea27a}
\end{exe}

Predicative constructions with nominals:

\begin{exe}
\ex Group denoting nominals: \emph{seem a big group}, \emph{be the organizing committee}, \emph{be a nice couple}.
\ex Relational nominals: \emph{be brothers}, \emph{sisters}, \emph{friends}.
\ex Nominals modified by collective adjectives: \emph{be numerous people}, \emph{similar students}, \emph{parallel lines}.
\ex Reciprocal possessives: \emph{be teachers of each other}, \emph{be admirers of one another}.
\label{org542b69f}
\end{exe}

\section{A type-theoretic approach to plural individuals}
\label{sec:orgd069a5c}

\subsection{Plural individuals}
\label{sec:orgf1fe5d5}

Fundamental hypothesis: a plural individual is a \emph{set of individuals}.

\begin{exe}
\ex John, Mary and Sue = \(\set{\mathbf{J}, \mathbf{M}, \mathbf{S}}\)
\ex The boys = \(\set{\mathbf{J},\mathbf{B},\mathbf{H}}\)
\label{orgc3db457}
\end{exe}

How do we model sets of individuals in the Simply-Typed \(\lambda\)-calculus? Recall, the denotation of an expression of type \(E \to T\) is a function \(f:\mathbf{Dom}_E \mapsto \mathbf{Dom}_E\).

A \emph{set} of individuals can be modelled as a function from individuals to truth-values, where members of the set are mapped to \textbf{true}, and non-members are mapped to \textbf{false}. This is called the \textbf{characteristic function} of the set.

Here's the characteristic function of the set of \emph{John}, \emph{Mary}, and \emph{Sue}:

\[\left[\begin{aligned}[c]
&\mathbf{J} &\mapsto \mathbf{true}\\
&\mathbf{M} &\mapsto \mathbf{true}\\
&\mathbf{S} &\mapsto \mathbf{true}\\
&\mathbf{B} &\mapsto \mathbf{false}\\
&\mathbf{H} &\mapsto \mathbf{false}\\
&\ldots
\end{aligned}\right]\]

Here's the characteristic function of the set of \emph{John}, \emph{Bill}, and \emph{Harry} (i.e., \emph{the boys}):

\[\left[\begin{aligned}[c]
&\mathbf{J} &\mapsto \mathbf{true}\\
&\mathbf{M} &\mapsto \mathbf{false}\\
&\mathbf{S} &\mapsto \mathbf{false}\\
&\mathbf{B} &\mapsto \mathbf{true}\\
&\mathbf{H} &\mapsto \mathbf{true}\\
&\ldots
\end{aligned}\right]\]

Our hypothesis will be that a \emph{group-denoting NP} is translated to an \emph{expression of type \(E \to T\)}

Since \emph{characteristic functions} are ways of encoding sets, we can also translate familiar set-theoretic notions of \emph{intersection} and \emph{union} into operations on functions. Look familiar?  

\begin{exe}
\ex \(P_{ET} \sqcap Q_{ET}= \lambda x\,.\,P(x) \wedge Q(x)\)
\ex \(P_{ET} \sqcup Q_{ET}= \lambda x\,.\,P(x) \vee Q(x)\)
\label{orgf060a17}
\end{exe}

If group-denoting NPs are of type \(E \to T\), there's an obvious way to encode the selectional requirements of strictly collective predicates - we can translate them as expressions of type \((E \to T) \to T\).

\begin{center}
\begin{tabular}{lll}
 & Singular/Distributive & Plural/Collective\\[0pt]
\hline
Individual & \(E\) & \(E \to T\)\\[0pt]
Predicate & \(E \to T\) & \((E \to T) \to T\)\\[0pt]
Quantifier & \((E \to T) \to T\) & \(((E \to T) \to T) \to T\)\\[0pt]
\end{tabular}
\end{center}

\subsection{Collective predication in practice}
\label{sec:orga3641d4}

Collective predicates denote \emph{higher-order functions} from functions \(f: \mathbf{Dom}_E \to \mathbf{Dom}_T\) to \(\mathbf{Dom}_T\).

What does this mean? \emph{met} takes a function \(P_{ET}\), and maps it to true, just in case the individuals that \(P\) maps to true met each other.

Imagine that the set of individuals is \(\set{a,b,c}\), and the only meetings that happened were between \(a,b\) and \(a,c\). The denotation of \(\mathbf{met}_{ET \to T}\) would be as follows:

\[\left[\begin{aligned}[c]
\left[\begin{aligned}[c]
&a \to \mathbf{t}\\
&b \to \mathbf{t}\\
&c \to \mathbf{t}
\end{aligned}\right] \to \mathbf{f}
\left[\begin{aligned}[c]
&a \to \mathbf{t}\\
&b \to \mathbf{t}\\
&c \to \mathbf{f}
\end{aligned}\right] \to \mathbf{t}
\left[\begin{aligned}[c]
&a \to \mathbf{t}\\
&b \to \mathbf{f}\\
&c \to \mathbf{t}
\end{aligned}\right] \to \mathbf{t}\\
\left[\begin{aligned}[c]
&a \to \mathbf{f}\\
&b \to \mathbf{t}\\
&c \to \mathbf{t}
\end{aligned}\right] \to \mathbf{f}
\left[\begin{aligned}[c]
&a \to \mathbf{t}\\
&b \to \mathbf{f}\\
&c \to \mathbf{f}
\end{aligned}\right] \to \mathbf{f}
\left[\begin{aligned}[c]
&a \to \mathbf{f}\\
&b \to \mathbf{t}\\
&c \to \mathbf{f}
\end{aligned}\right] \to \mathbf{f}\\
\left[\begin{aligned}[c]
&a \to \mathbf{f}\\
&b \to \mathbf{f}\\
&c \to \mathbf{t}
\end{aligned}\right] \to \mathbf{f}
\left[\begin{aligned}[c]
&a \to \mathbf{f}\\
&b \to \mathbf{f}\\
&c \to \mathbf{f}
\end{aligned}\right] \to \mathbf{f}
\end{aligned}\right]\]

Due to the equivalence between characteristic functions and sets, we can also write the denotation of a collective predicate as a \emph{set of sets}. This will often be much more convenient. I.e., the following set of sets encodes the same information.

\[\set{\set{a,b},\set{a,c}}\]

As a starting point, we'll assume that plural definites such as ``the boys'', and conjunctions of singular definites such as ``John, Mary, and Sue'', are translated as expressions of type \(E \to T\), and therefore denote sets of individuals.

\begin{exe}
\ex \(\mathbf{theBoys}_{ET}\)
\ex \(\mathbf{JohnMaryAndSue}_{ET}\)
\label{org99991fd}
\end{exe}

\section{The quantifier-collectivity connection and NP conjunction}
\label{sec:org2881e11}

\subsection{Generalized quantifiers and collective predicates}
\label{sec:org8131195}

On the approach to collective predicates outlined here, not that a collective predicate and a generalized quantifier are of the same type.

\begin{exe}
\ex \(\mathbf{meet}_{ET \to T} := \lambda X\,.\,\mathbf{meet}(X)\)
\ex \(\mathbf{noBoy}_{ET \to T} := \lambda X\,.\,\neg\exists x[\mathbf{boy}(x) \wedge X(x)]\)
\label{org97ba063}
\end{exe}

In set-talk:

\begin{exe}
\ex \(Set(\eval*{\mathbf{meet}}) = \set{X | \text{individuals in } X\text{ met}}\)
\ex \(Set(\eval*{\mathbf{noBoy}}) = \set{X | \text{there are no boys in }X }\)
\label{org1d45a63}
\end{exe}

For this reason, we (correctly in this case) predict that a quantificational NP and a collective predicate can't compose.

\begin{exe}
\ex *No boy met.
\label{org9b10039}
\end{exe}

\subsection{The problem of NP conjunction}
\label{sec:org367dfa3}

Now we're in a position to start thinking about how to account for the fact that conjoined NPs are compatible with collective predicates.

\begin{exe}
\ex Mary and Sue met.
\label{org6ce102b}
\end{exe}

Recall that conjunction can't even compose with \textbf{Mary} and \textbf{Sue} unless they undergo Montague lift.

Once we lift the NPs, doing generalized conjunction gives us back a generalized quantifier.

\begin{exe}
\ex \(\mathbf{Mary}^\uparrow \sqcap \mathbf{Sue}^\uparrow = \lambda P_{ET}\,.\,P(\mathbf{Mary}) \wedge P(\mathbf{Sue})\)
\label{org486f000}
\end{exe}

In set-talk:

\begin{exe}
\ex \(\set{P |\mathbf{Mary} \in P \text{ and }\mathbf{Sue} \in P}\)
\label{org6e83990}
\end{exe}

Note that aside from the typing problem, these aren't the right kind of sets to feed in as the argument of \textbf{meet} - they contain too many other individuals!

What we want is a type shifting function which takes the type \(ET \to T\) expression in (\ref{org486f000}), and gives back a \emph{generalized quantifier over collections} of type \((ET \to T) \to T\).

Winter posits two type-shifters to derive this (which we'll discuss in more detail next week).

\begin{itemize}
\item Minimum sort.
\item Existential raising.
\end{itemize}

\subsection{Minimum sort}
\label{sec:org7e978ef}

\textbf{Min} is an operator that takes a generalized quantifier \(Q\) of type \(ET \to T\), and gives back the \emph{minimal members} of \(Q\). This is defined formally as in (\ref{orgc2f0664}) (note that Winter gives a generalized version of ([[mmm\ref{orgc2f0664}), but we'll only need the formulation for quantifiers over individuals).

\begin{exe}
\ex \(\mathbf{Min}_{ETT \rightarrow ETT} := \lambda Q_{ETT}_{}\lambda A_{ET}\,.\, Q(A) \wedge \forall B \in Q[B \subseteq A \rightarrow B = A]\)
\label{orgc2f0664}
\end{exe}

\begin{itemize}
\item What is the result of applying \textbf{Min} to \emph{a boy}?
\item What is the result of applying \textbf{Min} to \emph{every boy}?
\item What is the result of applying \textbf{Min} to lifted \textbf{Mary}?
\item What is the result of applying \textbf{Min} to 'Mary and Sue'?
\end{itemize}

\begin{exe}
\ex \(\mathbf{Min}(\lambda P\,.\,P(\mathbf{Mary}) \wedge P(\mathbf{Sue})) = \set{\set{\mathbf{Mary},\mathbf{Sue}}}\)
\label{org5f07592}
\end{exe}

This still isn't of the right type to combine with a collective predicate, but we're getting closer.

\subsection{Existential raising}
\label{sec:orgc7f7afc}

\begin{exe}
\ex \(\mathbf{E}_{ETT \rightarrow ETT \to T} := \lambda A\,.\,\lambda P\,.\,\exists X[A(X) \wedge P(X)]\)
\label{org38878ae}
\end{exe}

\subsection{NP conjunction with collective predicates}
\label{sec:orgcba5a45}

\begin{exe}
\ex \(\mathbf{E}(\mathbf{Min}(\mathbf{Mary}^\uparrow \sqcap \mathbf{Sue}^\uparrow))(\mathbf{met})\)
\label{orgb670f0f}
\end{exe}

\subsection{Collectivity in Boolean domains}
\label{sec:org577b18e}

Why do we need to do minimum sort, and \emph{then} existentially raise? Since the minimum sort of a generalized conjunction of lifted individuals is a singleton set, why don't we define an operator which returns the unique minimal set of a quantifier.

Winter shows that this doesn't generalize to more complex examples involving conjunction \emph{and} disjunction.

\begin{exe}
\ex Mary and either Sue or John met.
\label{org0d14a68}
\end{exe}

Verify that the minimum sort has more than one member:

\begin{exe}
\ex \(\mathbf{Min}(\mathbf{Mary} \sqcap (\mathbf{Sue} \sqcup \mathbf{John}))\)
\label{org6a71835}
\end{exe}

\subsection{Super-generalized disjunction}
\label{sec:orgb2bdb49}

\begin{exe}
\ex Either Mary and Sue, or Sue and Bill met.
\label{orga5a4b54}
\end{exe}




\printbibliography
\end{document}