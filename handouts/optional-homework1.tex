% Created 2022-11-07 Mon 16:07
% Intended LaTeX compiler: pdflatex
\documentclass[letterpaper,parskip=half]{scrartcl}
\usepackage[utf8]{inputenc}
\usepackage[T1]{fontenc}
\usepackage{graphicx}
\usepackage{longtable}
\usepackage{wrapfig}
\usepackage{rotating}
\usepackage[normalem]{ulem}
\usepackage{amsmath}
\usepackage{amssymb}
\usepackage{capt-of}
\usepackage{hyperref}
\usepackage{braket}
\input{boilerplate}
\author{Patrick D. Elliott}
\date{\today}
\title{Optional homework}
\hypersetup{
 pdfauthor={Patrick D. Elliott},
 pdftitle={Optional homework},
 pdfkeywords={},
 pdfsubject={},
 pdfcreator={Emacs 28.2 (Org mode 9.5.5)}, 
 pdflang={English}}
\usepackage{biblatex}
\addbibresource{/home/patrl/repos/bibliography/master.bib}
\addbibresource{~/repos/bibliography/master.bib}
\begin{document}

\maketitle
\tableofcontents


\section{Set intersection and generalized conjunction}
\label{sec:org3e31a83}

The set characterized by a function \(f:\mathbf{Dom}_\sigma \mapsto \mathbf{Dom}_E\) is defined as follows:

\[Set(f) = \set{x \in \mathbf{Dom}_\sigma | f(x) = \mathbf{true}}\]

Generalized conjunction for predicates \(\sqcap_{ET}\) encodes set intersection.

Use the definition of generalized conjunction from the first handout to demonstrate that the following are equivalent:

\begin{itemize}
\item \(Set\eval*{\mathbf{sleep}_{ET}} \cap Set\eval*{\mathbf{laugh}_{ET}}\)
\item \(Set\eval*{\mathbf{sleep}_{ET} \sqcap_{ET} \mathbf{laugh}_{ET}}\)
\end{itemize}

\section{Generalized negation}
\label{sec:org6d52a83}

Use the definition of generalized conjunction in handout 1 as a basis for definining \emph{generalized negation}. Use boolean negation as a base:

\begin{itemize}
\item \(\mathbf{not}: T \to T\)
\end{itemize}

Show how this accounts for predicate negation:

\begin{itemize}
\item ``John is not tall''
\end{itemize}

Assume the translation: \(\mathbf{not}(\mathbf{tall})(\mathbf{John})\)


\printbibliography
\end{document}